\documentclass{article}
\usepackage[utf8]{inputenc}
\usepackage{amsmath}
\usepackage{graphicx}
\usepackage{subcaption}
\usepackage{caption}
\usepackage{color}
\usepackage[a4paper, total={6in, 8in}]{geometry}
\usepackage{tikz}
\usepackage{listings}
\usepackage{amsthm}
\usepackage{chemformula}
\pagenumbering{arabic}
\usepackage{rotating}
\usepackage{hyperref}
\usepackage{float}
%taxing in colour
\definecolor{mygreen}{rgb}{0,0.6,0} % R G B values
\definecolor{mygray}{rgb}{0.5,0.5,0.5}
\definecolor{mymauve}{rgb}{0.58,0,0.82}

\author{}
\date{}
\linespread{1.2}
\begin{document}
\begin{titlepage}
    \begin{center}
%        \begin{figure}[h]
 %               \includegraphics[scale=0.15]{Image/ICL logo.png}
%            \end{subfigure}
 %       \end{figure}        
        \vspace*{3cm}
        \huge
        \textbf{Pore scale imaging, analysis, and data-driven pore-scale modelling}\\
        \textbf{\Large Report 1: }\\
        \vspace*{1cm}
        
        \vspace*{2cm}
        \author[{\textbf{Sepideh Goodarzi}}\\
        
        \large
        \vfill
        \centering\text{Supervisors: Prof. Martin Blunt and Dr. Branko Bijeljic }\\
        \vspace*{1cm}
        \centering\text{May 2021}\\
        \centering\text{Imperial College London}\\
        \centering\text{Department of Earth Science and Engineering}
    \end{center}
\end{titlepage}
% lstset for coding
\lstset{
 backgroundcolor=\color{white},   % choose the background color; you must add \usepackage{color} or \usepackage{xcolor}
 basicstyle=\footnotesize,        % the size of the fonts that are used for the code
 breakatwhitespace=false,         % sets if automatic breaks should only happen at whitespace
 breaklines=true,                 % sets automatic line breaking
 captionpos=b,                    % sets the caption-position to bottom
 commentstyle=\color{mygreen},    % comment style
 deletekeywords={...},            % if you want to delete keywords from the given language
 escapeinside={\%*}{*)},          % if you want to add LaTeX within your code
 extendedchars=true,              % lets you use non-ASCII characters; for 8-bits encodings only, does not work with UTF-8
 frame=single,	                   % adds a frame around the code
 keepspaces=true,                 % keeps spaces in text, useful for keeping indentation of code (possibly needs columns=flexible)
  keywordstyle=\color{blue},       % keyword style
  language=Python,                 % the language of the code
  otherkeywords={*,np, plt},            % if you want to add more keywords to the set
  numbers=left,                    % where to put the line-numbers; possible values are (none, left, right)
  numbersep=5pt,                   % how far the line-numbers are from the code
  numberstyle=\tiny\color{black}, % the style that is used for the line-numbers
  rulecolor=\color{mygray},         % if not set, the frame-color may be changed on line-breaks within not-black text (e.g. comments (green here))
  showspaces=false,                % show spaces everywhere adding particular underscores; it overrides 'showstringspaces'
  showstringspaces=false,          % underline spaces within strings only
  showtabs=true,                  % show tabs within strings adding particular underscores
  stepnumber=2,                    % the step between two line-numbers. If it's 1, each line will be numbered
  stringstyle=\color{mymauve},     % string literal style
  tabsize=4,	                   % sets default tabsize to 4 spaces
  title=\lstname                   % show the filename of files included with \lstinputlisting; also try caption instead of title
}
\pagenumbering{roman}
%\input{Abstract}
\cleardoublepage
\pagenumbering{roman}
% \setcounter{page}{2}
\tableofcontents
\thispagestyle{empty}

\cleardoublepage
%\setcounter{page}{1}
\pagenumbering{arabic}

\newpage
\section{Disconnected Gas Transport in Steady-State Three-Phase Flow }
reference: \\
abstract main paper
\subsection{Martin}\label{martin}
refrence: M. J. Blunt, Multiphase flow in permeable media: A pore-scale perspective. Cambridge University Press, 2017.\\
Quantifying three-phase flow is essential for the design of reservoir models that simulate the behaviour of gas, oil and water in geological systems

\subsection{Ali}
reference: \\

\section{2-phase}
refrence: \\

\subsection{Yihuai}
numbers coming from \ref{martin}




\end{document}